%
% WDCPAPER.TEX
%
% Paper sent to the World Data Center describing the catalogues of
% constellation boundary data.
%
% A C Davenhall (ROE).                 1/8/88.
% A C Davenhall (ROE).  {Revised}      20/2/90.
%
\documentstyle[12pt]{article}
\title{A Catalogue of Constellation Boundary Data}
\author{A.C. Davenhall \thanks{Royal Observatory, Blackford Hill,
           Edinburgh, EH9 3HJ, Scotland.}  \\
        S.K. Leggett \thanks{Dept. of Astronomy, University of
           Edinburgh, Blackford Hill, Edinburgh, EH9 3HJ, Scotland.}
           \thanks{Present address, US Naval Observatory, Flagstaff,
           Arizona, USA.}  }
\date{February 1990}

\setcounter{secnumdepth}{3}
\setcounter{tocdepth}{3}
%
% Start document.
%
\begin{document}

\maketitle

\begin{abstract}
A computer readable catalogue of constellation boundary data is
presented in a form suitable for the construction of star charts and
atlases. Two files are available, one for equator and equinox 1875 and
the other for equator and equinox 2000.
\end{abstract}

\section{INTRODUCTION}

The definitive list of constellation boundary data was published by Delporte
(1930). A computer-readable catalogue compiled from these data has been produced
by Roman (1987) and is available from the International Network of Astronomical
Data Centers (catalogue number 6042). However, the format of this catalogue has
been chosen to facilitate locating the constellation in which a given object
lies and is unsuited to the construction of star charts and atlases. The present
catalogue was also constructed from Delporte's lists, but the files are arranged
in a format suitable for constructing star charts and atlases.

Two files are available in a computer-readable form. The first gives the
boundaries as delineated by Delporte, in their original orientation, 1875. The
second file gives the boundaries computed for equator and equinox 2000. In
equator and equinox 1875 the lines joining the corners of the constellations
were great circles of right ascension and parallels of declination. However,
precession to equator and equinox 2000 distorts the boundaries so that they no
longer lie along lines of constant right ascension or declination. Thus, prior
to precession to equator and equinox 2000, points were interpolated at one
degree intervals along the boundaries in order that they should continue to
enclose the same area of sky (and thus the same set of stars).

\section{THE CATALOGUE}

Brief details of the two files of the catalogue are given below.

\subsection{Equator and equinox 1875 file}

The data in this file was taken directly from Delporte with the exception that
several extra points were added to the constellation Octans (which covers the
South Pole) in order to facilitate plotting in some projections. These extra
points `traverse' the line of 0hr right ascension to the Pole and `return' at
24hr.

The file consists of the right ascension and declination of every corner of
every constellation. Each record consists of a single point along the
constellation boundary and contains the right ascension and declination of the
point and an abbreviation identifying the constellation of which it is a part.
The abbreviations used to identify each constellation are taken from Delporte,
and for completeness they are listed together with the full name in Table 1. All
the abbreviations consist of three characters except those for Serpens Caput and
Serpens Cauda which contain four. All records referring to a given constellation
occur contiguously in the file. Successive records correspond to successive
points along the constellation boundary. The last point in the constellation
should be joined to the first. Table 2 gives details of the structure of each
record.

The following fragment of FORTRAN could be used to read a single record from the
file

\begin{verbatim}

      REAL RA, DEC
      CHARACTER*4 CONNAM

      .
      .

      READ(13, 1000) RA, DEC, CONNAM
 1000 FORMAT(F8.5, 1X, F9.5, 1X, A4)

      .
      .

\end{verbatim}

\subsection{Equator and equinox 2000 file}

In equator and equinox 1875, lines joining the corners of constellations are
either great circles of right ascension or parallels of declination. When the
coordinates are converted to equator and equinox 2000, the boundaries are
distorted by precession so that they no longer lie along lines of constant right
ascension or declination. In order that the new boundaries should enclose the
same area of sky (and thus include the same stars) it is necessary that points
should be interpolated along the boundary prior to calculating the precession.
Thus the orientation 2000 file was generated from the orientation 1875 file with
points interpolated at one-degree intervals along the boundaries. When the file
is used to construct a chart, a smooth curve should be drawn through the
interpolated points (but obviously not through those original points that define
corners).

Records in the file are ordered into constellations and successive corners
within each constellation in a similar fashion to the orientation 1875 file.
Table 3 lists the details of the structure of each record. Like the orientation
1875 file, each record contains the right ascension and declination of a point
together with an abbreviation for the constellation to which it belongs. However
an additional field indicates whether the point is an original point taken
directly from Delporte (code `O') or an interpolated point (code `I'). It should
be noted that original points can be points along a meridian or parallel where
three constellations meet as well as corners.

The following fragment of FORTRAN could be used to read a single record from the
file

\begin{verbatim}

      REAL RA, DEC
      CHARACTER*4 CONNAM
      CHARACTER*1 TYPE

      .
      .

      READ(13, 1001) RA, DEC, CONNAM, TYPE
 1001 FORMAT(F10.7, 1X, F11.7, 1X, A4, 1X A1)

      .
      .

\end{verbatim}

\section{DISCUSSION}

The present catalogue of constellation boundary data is complementary to that of
Roman (1987). Roman's catalogue should be used to determine which constellation
an object lies in. The present catalogue is more suited to the construction of
star charts and atlases.

\section{ACKNOWLEDGEMENTS}

We are grateful to Mr D.A. Pickup for useful discussions about the format of
Delporte's lists and the effect of precession on the shape of the constellation
boundaries and to Dr W.H. Warren Jr for several useful suggestions. All
computations were carried out on the Starlink VAX 11/780 at the Royal
Observatory Edinburgh, and the catalogue was precessed using the SCAR
astronomical database software available within the Starlink software
collection. This work was carried out as part of a contract with the Edinburgh
publishers John Bartholomew and Son Ltd. to provide data for a revised edition
of `Norton's Star Atlas'.

\section{REFERENCES}

Delporte, E. 1930, {\it Delimitation Scientifique des Constellations}
(Cambridge: Cambridge University Press).
\\[2.0ex]
Roman, N.G. 1987, {\it Publ. Astron. Soc. Pac.} \underline{99},
pp695-699.

%
% Constellation names and abbreviations.
%

\begin{table}

\begin{tabular}{|l|l|l|l|l|}\hline
Abbreviation  &  Constellation Name & &  Abbreviation  &  Constellation Name \\ \hline
AND  &  Andromeda      & &  DOR  &  Dorado  \\
ANT  &  Antila         & &  DRA  &  Draco  \\
APS  &  Apus           & &  EQU  &  Equuleus  \\
AQR  &  Aquarius       & &  ERI  &  Eridanus  \\
AQL  &  Aquila         & &  FOR  &  Fornax  \\
ARA  &  Ara            & &  GEM  &  Gemini  \\
ARI  &  Aries          & &  GRU  &  Grus  \\
AUR  &  Auriga         & &  HER  &  Hercules  \\
BOO  &  Bootes         & &  HOR  &  Horologium  \\
CAE  &  Caelum         & &  HYA  &  Hydra  \\
CAM  &  Camelopardis   & &  HYI  &  Hydrus  \\
CNC  &  Cancer         & &  IND  &  Indus  \\
CVN  &  Canes Venatici & &  LAC  &  Lacerta  \\
CMA  &  Canis Major    & &  LEO  &  Leo  \\
CMI  &  Canis Minor    & &  LMI  &  Leo Minor  \\
CAP  &  Capricornus    & &  LEP  &  Lepus  \\
CAR  &  Carina         & &  LIB  &  Libra  \\
CAS  &  Cassiopeia     & &  LUP  &  Lupus  \\
CEN  &  Centaurus      & &  LYN  &  Lynx  \\
CEP  &  Cepheus        & &  LYR  &  Lyra  \\
CET  &  Cetus          & &  MEN  &  Mensa  \\
CHA  &  Chamaeleon     & &  MIC  &  Microscopium  \\
CIR  &  Circinus       & &  MON  &  Monoceros  \\
COL  &  Columba        & &  MUS  &  Musca  \\
COM  &  Coma Berenices & &  NOR  &  Norma  \\
CRA  &  Corona Australis & &  OCT  &  Octans  \\
CRB  &  Corona Borealis  & &  OPH  &  Ophiuchus  \\
CRV  &  Corvus         & & ORI  &  Orion  \\
CRT  &  Crater         & &  PAV  &  Pavo  \\
CRU  &  Crux           & &  PEG  &  Pegasus  \\
CYG  &  Cygnus         & &  PER  &  Perseus  \\
DEL  &  Delphinus      & &  PHE  &  Phoenix  \\ \hline
\end{tabular}

\begin{center}
Table 1: Abbreviations for constellation names.
\end{center}

\end{table}

\begin{table}

\begin{tabular}{|l|l|l|l|l|}\hline
Abbreviation  &  Constellation Name & &  Abbreviation  &  Constellation Name \\ \hline

PIC  &  Pictor         & &  SEX  &  Sextans  \\
PSC  &  Pisces         & &  TAU  &  Taurus  \\
PSA  &  Pisces Austrinus & &  TEL  &  Telescopium  \\
PUP  &  Puppis         & &  TRI  &  Triangulum  \\
PYX  &  Pyxis          & &  TRA  &  Triangulum Australe  \\
RET  &  Reticulum      & &  TUC  &  Tucana  \\
SGE  &  Sagitta        & &  UMA  &  Ursa Major  \\
SGR  &  Sagittarius    & &  UMI  &  Ursa Minor  \\
SCO  &  Scorpius       & &  VEL  &  Vela  \\
SCL  &  Sculptor       & &  VIR  &  Virgo  \\
SCT  &  Scutum         & &  VOL  &  Volans  \\
SER1 &  Serpens Caput  & &  VUL  &  Vulpecula  \\
SER2 &  Serpens Cauda  & &  & \\ \hline
\end{tabular}

\begin{center}
Table 1 (continued): Abbreviations for constellation names.
\end{center}

\end{table}

%
%  details for equator and equinox 1985 file.
%

\begin{table}

\begin{tabular}{|l|l|}\hline
Number of records           &  1566  \\
Record size (bytes)         &  25    \\ \hline
\end{tabular}

\begin{tabular}{|l|l|l|l|}\hline
Field           & Units           & Starting byte &  Format  \\ \hline
Right ascension & Decimal hours   & 1             &  F8.5    \\
Declination     & Decimal degrees & 10            &  F9.5    \\
Constellation   &                 & 20            &  A4      \\
Type of point   &                 & 25            &  A1      \\ \hline
\end{tabular}

\begin{center}
Table 2: Format of equator and equinox 1875 file.
\end{center}

\end{table}


%
% details of equator and equinox 2000 file.
%

\begin{table}

\begin{tabular}{|l|l|}\hline
Number of records                &  13422  \\
Record size (bytes)              &  29     \\ \hline
\end{tabular}

\begin{tabular}{|l|l|l|l|}\hline
Field           & Units           & Starting byte &  Format  \\ \hline
Right ascension & Decimal hours   & 1             &  F10.7   \\
Declination     & Decimal degrees & 12            &  F11.7   \\
Constellation   &                 & 24            &  A4      \\
Type of point   &                 & 29            &  A1      \\ \hline
\end{tabular}

\begin{center}
Table 3: Format of equator and equinox 2000 file.
\end{center}

\end{table}

\end{document}
